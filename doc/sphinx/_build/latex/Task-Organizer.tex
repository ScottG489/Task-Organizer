% Generated by Sphinx.
\def\sphinxdocclass{report}
\documentclass[letterpaper,10pt,english]{sphinxmanual}
\usepackage[utf8]{inputenc}
\DeclareUnicodeCharacter{00A0}{\nobreakspace}
\usepackage[T1]{fontenc}
\usepackage{babel}
\usepackage{times}
\usepackage[Bjarne]{fncychap}
\usepackage{longtable}
\usepackage{sphinx}


\title{Task-Organizer Documentation}
\date{September 19, 2011}
\release{0.0.1}
\author{Scott Giminiani}
\newcommand{\sphinxlogo}{}
\renewcommand{\releasename}{Release}
\makeindex

\makeatletter
\def\PYG@reset{\let\PYG@it=\relax \let\PYG@bf=\relax%
    \let\PYG@ul=\relax \let\PYG@tc=\relax%
    \let\PYG@bc=\relax \let\PYG@ff=\relax}
\def\PYG@tok#1{\csname PYG@tok@#1\endcsname}
\def\PYG@toks#1+{\ifx\relax#1\empty\else%
    \PYG@tok{#1}\expandafter\PYG@toks\fi}
\def\PYG@do#1{\PYG@bc{\PYG@tc{\PYG@ul{%
    \PYG@it{\PYG@bf{\PYG@ff{#1}}}}}}}
\def\PYG#1#2{\PYG@reset\PYG@toks#1+\relax+\PYG@do{#2}}

\def\PYG@tok@gd{\def\PYG@tc##1{\textcolor[rgb]{0.63,0.00,0.00}{##1}}}
\def\PYG@tok@gu{\let\PYG@bf=\textbf\def\PYG@tc##1{\textcolor[rgb]{0.50,0.00,0.50}{##1}}}
\def\PYG@tok@gt{\def\PYG@tc##1{\textcolor[rgb]{0.00,0.25,0.82}{##1}}}
\def\PYG@tok@gs{\let\PYG@bf=\textbf}
\def\PYG@tok@gr{\def\PYG@tc##1{\textcolor[rgb]{1.00,0.00,0.00}{##1}}}
\def\PYG@tok@cm{\let\PYG@it=\textit\def\PYG@tc##1{\textcolor[rgb]{0.25,0.50,0.56}{##1}}}
\def\PYG@tok@vg{\def\PYG@tc##1{\textcolor[rgb]{0.73,0.38,0.84}{##1}}}
\def\PYG@tok@m{\def\PYG@tc##1{\textcolor[rgb]{0.13,0.50,0.31}{##1}}}
\def\PYG@tok@mh{\def\PYG@tc##1{\textcolor[rgb]{0.13,0.50,0.31}{##1}}}
\def\PYG@tok@cs{\def\PYG@tc##1{\textcolor[rgb]{0.25,0.50,0.56}{##1}}\def\PYG@bc##1{\colorbox[rgb]{1.00,0.94,0.94}{##1}}}
\def\PYG@tok@ge{\let\PYG@it=\textit}
\def\PYG@tok@vc{\def\PYG@tc##1{\textcolor[rgb]{0.73,0.38,0.84}{##1}}}
\def\PYG@tok@il{\def\PYG@tc##1{\textcolor[rgb]{0.13,0.50,0.31}{##1}}}
\def\PYG@tok@go{\def\PYG@tc##1{\textcolor[rgb]{0.19,0.19,0.19}{##1}}}
\def\PYG@tok@cp{\def\PYG@tc##1{\textcolor[rgb]{0.00,0.44,0.13}{##1}}}
\def\PYG@tok@gi{\def\PYG@tc##1{\textcolor[rgb]{0.00,0.63,0.00}{##1}}}
\def\PYG@tok@gh{\let\PYG@bf=\textbf\def\PYG@tc##1{\textcolor[rgb]{0.00,0.00,0.50}{##1}}}
\def\PYG@tok@ni{\let\PYG@bf=\textbf\def\PYG@tc##1{\textcolor[rgb]{0.84,0.33,0.22}{##1}}}
\def\PYG@tok@nl{\let\PYG@bf=\textbf\def\PYG@tc##1{\textcolor[rgb]{0.00,0.13,0.44}{##1}}}
\def\PYG@tok@nn{\let\PYG@bf=\textbf\def\PYG@tc##1{\textcolor[rgb]{0.05,0.52,0.71}{##1}}}
\def\PYG@tok@no{\def\PYG@tc##1{\textcolor[rgb]{0.38,0.68,0.84}{##1}}}
\def\PYG@tok@na{\def\PYG@tc##1{\textcolor[rgb]{0.25,0.44,0.63}{##1}}}
\def\PYG@tok@nb{\def\PYG@tc##1{\textcolor[rgb]{0.00,0.44,0.13}{##1}}}
\def\PYG@tok@nc{\let\PYG@bf=\textbf\def\PYG@tc##1{\textcolor[rgb]{0.05,0.52,0.71}{##1}}}
\def\PYG@tok@nd{\let\PYG@bf=\textbf\def\PYG@tc##1{\textcolor[rgb]{0.33,0.33,0.33}{##1}}}
\def\PYG@tok@ne{\def\PYG@tc##1{\textcolor[rgb]{0.00,0.44,0.13}{##1}}}
\def\PYG@tok@nf{\def\PYG@tc##1{\textcolor[rgb]{0.02,0.16,0.49}{##1}}}
\def\PYG@tok@si{\let\PYG@it=\textit\def\PYG@tc##1{\textcolor[rgb]{0.44,0.63,0.82}{##1}}}
\def\PYG@tok@s2{\def\PYG@tc##1{\textcolor[rgb]{0.25,0.44,0.63}{##1}}}
\def\PYG@tok@vi{\def\PYG@tc##1{\textcolor[rgb]{0.73,0.38,0.84}{##1}}}
\def\PYG@tok@nt{\let\PYG@bf=\textbf\def\PYG@tc##1{\textcolor[rgb]{0.02,0.16,0.45}{##1}}}
\def\PYG@tok@nv{\def\PYG@tc##1{\textcolor[rgb]{0.73,0.38,0.84}{##1}}}
\def\PYG@tok@s1{\def\PYG@tc##1{\textcolor[rgb]{0.25,0.44,0.63}{##1}}}
\def\PYG@tok@gp{\let\PYG@bf=\textbf\def\PYG@tc##1{\textcolor[rgb]{0.78,0.36,0.04}{##1}}}
\def\PYG@tok@sh{\def\PYG@tc##1{\textcolor[rgb]{0.25,0.44,0.63}{##1}}}
\def\PYG@tok@ow{\let\PYG@bf=\textbf\def\PYG@tc##1{\textcolor[rgb]{0.00,0.44,0.13}{##1}}}
\def\PYG@tok@sx{\def\PYG@tc##1{\textcolor[rgb]{0.78,0.36,0.04}{##1}}}
\def\PYG@tok@bp{\def\PYG@tc##1{\textcolor[rgb]{0.00,0.44,0.13}{##1}}}
\def\PYG@tok@c1{\let\PYG@it=\textit\def\PYG@tc##1{\textcolor[rgb]{0.25,0.50,0.56}{##1}}}
\def\PYG@tok@kc{\let\PYG@bf=\textbf\def\PYG@tc##1{\textcolor[rgb]{0.00,0.44,0.13}{##1}}}
\def\PYG@tok@c{\let\PYG@it=\textit\def\PYG@tc##1{\textcolor[rgb]{0.25,0.50,0.56}{##1}}}
\def\PYG@tok@mf{\def\PYG@tc##1{\textcolor[rgb]{0.13,0.50,0.31}{##1}}}
\def\PYG@tok@err{\def\PYG@bc##1{\fcolorbox[rgb]{1.00,0.00,0.00}{1,1,1}{##1}}}
\def\PYG@tok@kd{\let\PYG@bf=\textbf\def\PYG@tc##1{\textcolor[rgb]{0.00,0.44,0.13}{##1}}}
\def\PYG@tok@ss{\def\PYG@tc##1{\textcolor[rgb]{0.32,0.47,0.09}{##1}}}
\def\PYG@tok@sr{\def\PYG@tc##1{\textcolor[rgb]{0.14,0.33,0.53}{##1}}}
\def\PYG@tok@mo{\def\PYG@tc##1{\textcolor[rgb]{0.13,0.50,0.31}{##1}}}
\def\PYG@tok@mi{\def\PYG@tc##1{\textcolor[rgb]{0.13,0.50,0.31}{##1}}}
\def\PYG@tok@kn{\let\PYG@bf=\textbf\def\PYG@tc##1{\textcolor[rgb]{0.00,0.44,0.13}{##1}}}
\def\PYG@tok@o{\def\PYG@tc##1{\textcolor[rgb]{0.40,0.40,0.40}{##1}}}
\def\PYG@tok@kr{\let\PYG@bf=\textbf\def\PYG@tc##1{\textcolor[rgb]{0.00,0.44,0.13}{##1}}}
\def\PYG@tok@s{\def\PYG@tc##1{\textcolor[rgb]{0.25,0.44,0.63}{##1}}}
\def\PYG@tok@kp{\def\PYG@tc##1{\textcolor[rgb]{0.00,0.44,0.13}{##1}}}
\def\PYG@tok@w{\def\PYG@tc##1{\textcolor[rgb]{0.73,0.73,0.73}{##1}}}
\def\PYG@tok@kt{\def\PYG@tc##1{\textcolor[rgb]{0.56,0.13,0.00}{##1}}}
\def\PYG@tok@sc{\def\PYG@tc##1{\textcolor[rgb]{0.25,0.44,0.63}{##1}}}
\def\PYG@tok@sb{\def\PYG@tc##1{\textcolor[rgb]{0.25,0.44,0.63}{##1}}}
\def\PYG@tok@k{\let\PYG@bf=\textbf\def\PYG@tc##1{\textcolor[rgb]{0.00,0.44,0.13}{##1}}}
\def\PYG@tok@se{\let\PYG@bf=\textbf\def\PYG@tc##1{\textcolor[rgb]{0.25,0.44,0.63}{##1}}}
\def\PYG@tok@sd{\let\PYG@it=\textit\def\PYG@tc##1{\textcolor[rgb]{0.25,0.44,0.63}{##1}}}

\def\PYGZbs{\char`\\}
\def\PYGZus{\char`\_}
\def\PYGZob{\char`\{}
\def\PYGZcb{\char`\}}
\def\PYGZca{\char`\^}
\def\PYGZsh{\char`\#}
\def\PYGZpc{\char`\%}
\def\PYGZdl{\char`\$}
\def\PYGZti{\char`\~}
% for compatibility with earlier versions
\def\PYGZat{@}
\def\PYGZlb{[}
\def\PYGZrb{]}
\makeatother

\begin{document}

\maketitle
\tableofcontents
\phantomsection\label{index::doc}


Contents:


\chapter{5. taskstorage}
\label{index:module-cliparser}\label{index:welcome-to-task-organizer-s-documentation}\label{index:taskstorage}
\index{cliparser (module)}
The docstring for a module should generally list the classes, exceptions and functions (and any other objects) that are exported by the module, with a one-line summary of each. (These summaries generally give less detail than the summary line in the object's docstring.) The docstring for a package (i.e., the docstring of the package's \_\_init\_\_.py module) should also list the modules and subpackages exported by the package.


\chapter{5. taskstorage}
\label{index:id1}\phantomsection\label{index:module-storage}
\index{storage (module)}
Facilitate Tasks in persistant storage.
\begin{description}
\item[{Public Classes:}] \leavevmode
FileStorage
SQLiteStorage
GTaskStorage
StorageFactory

\end{description}

Provides an interface to persist Task objects in differet storage mediums.

\index{FileStorage (class in storage)}

\begin{fulllineitems}
\phantomsection\label{index:storage.FileStorage}\pysiglinewithargsret{\strong{class }\code{storage.}\bfcode{FileStorage}}{\emph{task\_filename='taskfile'}, \emph{key\_filename='keyfile'}}{}
Interface for storing Tasks to a file.
\begin{description}
\item[{Kwargs/Instance Vars:}] \leavevmode
task\_filename (str): Name of file in which to store the Task list.
key\_filename (str): Name of file in which to store the next key.

\item[{Public methods:}] \leavevmode
add(task\_item)
find(key = None)
get\_all()
update(task\_item)
delete(key)
search(search\_task)

\end{description}

Reads and writes Task objects from a file as a single list.

\index{add() (storage.FileStorage method)}

\begin{fulllineitems}
\phantomsection\label{index:storage.FileStorage.add}\pysiglinewithargsret{\bfcode{add}}{\emph{task\_item}}{}
Add a Task to the file storage.
\begin{description}
\item[{Args:}] \leavevmode
task\_item (Task): The Task object to be added to storage.

\item[{Returns:}] \leavevmode
task\_item.key (int): Newly added Task's key.

\end{description}

Raises:

The Task object is given a key and appended to the list of Tasks in the file.

\end{fulllineitems}


\index{delete() (storage.FileStorage method)}

\begin{fulllineitems}
\phantomsection\label{index:storage.FileStorage.delete}\pysiglinewithargsret{\bfcode{delete}}{\emph{key}}{}
Delete an existing Task in the file storage.
\begin{description}
\item[{Args:}] \leavevmode
key (int): The key for the desired Task object to delete.

\item[{Returns:}] \leavevmode
key\_match (int): Task's key that was deleted in storage.

\end{description}

Raises:

Using the given key, iterate through the Task list and delete the 
matching Task. If none is found, nothing is deleted and return None.

\end{fulllineitems}


\index{find() (storage.FileStorage method)}

\begin{fulllineitems}
\phantomsection\label{index:storage.FileStorage.find}\pysiglinewithargsret{\bfcode{find}}{\emph{key=None}}{}
Return a Task given it's key.
\begin{description}
\item[{Args:}] \leavevmode
key (int): The key for the desired Task object.

\item[{Returns:}] \leavevmode
task\_item (Task): Task with matching key.

\end{description}

Raises:

Using the given key, iterate through the Task list and return the Task
with matching key. If none is found return None.

\end{fulllineitems}


\index{get\_all() (storage.FileStorage method)}

\begin{fulllineitems}
\phantomsection\label{index:storage.FileStorage.get_all}\pysiglinewithargsret{\bfcode{get\_all}}{}{}
Return a list of all Tasks.
\begin{description}
\item[{Returns:}] \leavevmode
task\_list (Task{[}{]}): List of every task in storage.

\end{description}

Raises:

\end{fulllineitems}


\index{search() (storage.FileStorage method)}

\begin{fulllineitems}
\phantomsection\label{index:storage.FileStorage.search}\pysiglinewithargsret{\bfcode{search}}{\emph{search\_task}}{}
Return a Task list given a search Task.
\begin{description}
\item[{Args:}] \leavevmode
search\_task (Task): The Task to be used for searching.

\item[{Returns:}] \leavevmode
task\_search\_list (Task{[}{]}): List of Tasks matching search criteria.

\end{description}

Raises:

Using the given search Task, iterate through the Task list and append
matching Tasks to a Task list then return this list. If none matches,
return None.

\end{fulllineitems}


\index{update() (storage.FileStorage method)}

\begin{fulllineitems}
\phantomsection\label{index:storage.FileStorage.update}\pysiglinewithargsret{\bfcode{update}}{\emph{task\_item}}{}
Update an existing Task in the file storage.
\begin{description}
\item[{Args:}] \leavevmode
task\_item (Task): The Task object to be updated.

\item[{Returns:}] \leavevmode
key\_match (int): Task's key that was updating in storage.

\end{description}

Raises:

Using the given Task's key, iterate through the Task list to find a
matching key, replace the matching Task with the given Task, and
return the old Task. If none is found, update nothing and return None.

\end{fulllineitems}


\end{fulllineitems}


\index{GTaskStorage (class in storage)}

\begin{fulllineitems}
\phantomsection\label{index:storage.GTaskStorage}\pysigline{\strong{class }\code{storage.}\bfcode{GTaskStorage}}{}
Interface for storing Tasks to Google Tasks.
\begin{description}
\item[{Public methods:}] \leavevmode
add(task\_item)
find(key = None)
get\_all()
update(task\_item)
delete(key)
search(search\_task)

\end{description}

Reads and writes Task from Google Tasks. Task objects are transformed
to and from Google's task dictionaries.

\index{add() (storage.GTaskStorage method)}

\begin{fulllineitems}
\phantomsection\label{index:storage.GTaskStorage.add}\pysiglinewithargsret{\bfcode{add}}{\emph{task\_item}}{}
Add a Task to the GTask storage.
\begin{description}
\item[{Args:}] \leavevmode
task\_item (Task): The Task object to be added to storage.

\item[{Returns:}] \leavevmode
task\_item.key (int): Newly added Task's key.

\end{description}

Raises:

The Task object is added to storage and given a key.

\end{fulllineitems}


\index{delete() (storage.GTaskStorage method)}

\begin{fulllineitems}
\phantomsection\label{index:storage.GTaskStorage.delete}\pysiglinewithargsret{\bfcode{delete}}{\emph{key}}{}
Delete an existing Task in the GTask storage.
\begin{description}
\item[{Args:}] \leavevmode
key (int): The key for the desired Task object to delete.

\item[{Returns:}] \leavevmode
key\_match (int): Task's key that was deleted in storage.

\end{description}

Raises:

Using the given key, delete the matching Task. If none is found,
nothing is deleted and return None.

\end{fulllineitems}


\index{find() (storage.GTaskStorage method)}

\begin{fulllineitems}
\phantomsection\label{index:storage.GTaskStorage.find}\pysiglinewithargsret{\bfcode{find}}{\emph{key=None}}{}
Return a Task given it's key.
\begin{description}
\item[{Args:}] \leavevmode
key (int): The key for the desired Task object.

\item[{Returns:}] \leavevmode
task\_item (Task): Task with matching key.

\end{description}

Raises:

Using the given key, return the Task with the matching key. If none
is found return None.

\end{fulllineitems}


\index{get\_all() (storage.GTaskStorage method)}

\begin{fulllineitems}
\phantomsection\label{index:storage.GTaskStorage.get_all}\pysiglinewithargsret{\bfcode{get\_all}}{}{}
Return a list of all Tasks.

\end{fulllineitems}


\index{search() (storage.GTaskStorage method)}

\begin{fulllineitems}
\phantomsection\label{index:storage.GTaskStorage.search}\pysiglinewithargsret{\bfcode{search}}{\emph{search\_task}}{}
Return a Task list given a search Task.
\begin{description}
\item[{Args:}] \leavevmode
search\_task (Task): The Task to be used for searching.

\item[{Returns:}] \leavevmode
task\_search\_list (Task{[}{]}): List of Tasks matching search criteria.

\end{description}

Raises:

Using the given search Task, iterate through the Task list and append
matching Tasks to a Task list then return this list. If none matches,
return None.

\end{fulllineitems}


\index{update() (storage.GTaskStorage method)}

\begin{fulllineitems}
\phantomsection\label{index:storage.GTaskStorage.update}\pysiglinewithargsret{\bfcode{update}}{\emph{task\_item}}{}
Update an existing Task in the GTask storage.
\begin{description}
\item[{Args:}] \leavevmode
task\_item (Task): The Task object to be updated.

\item[{Returns:}] \leavevmode
key\_match (int): Task's key that was updating in storage.

\end{description}

Raises:

Using the given Task's key, find the Task with a matching key and
replace it with the given Task. Then return the old Task. If none
is found, updating nothing and return None.

\end{fulllineitems}


\end{fulllineitems}


\index{SQLiteStorage (class in storage)}

\begin{fulllineitems}
\phantomsection\label{index:storage.SQLiteStorage}\pysiglinewithargsret{\strong{class }\code{storage.}\bfcode{SQLiteStorage}}{\emph{task\_dbname='taskdb'}}{}
Interface for storing Tasks to a SQLite database.
\begin{description}
\item[{Kwargs/Instance Vars:}] \leavevmode
task\_dbname (str): Name of database/file in which to store Tasks.

\item[{Public methods:}] \leavevmode
add(task\_item)
find(key = None)
get\_all()
update(task\_item)
delete(key)
search(search\_task)

\end{description}

Reads and writes Task objects from a sqlite database file. Tasks are
stored in a table whos columns coincide with the Task's attributes.

\index{add() (storage.SQLiteStorage method)}

\begin{fulllineitems}
\phantomsection\label{index:storage.SQLiteStorage.add}\pysiglinewithargsret{\bfcode{add}}{\emph{task\_item}}{}
Add a Task to the database storage.
\begin{description}
\item[{Args:}] \leavevmode
task\_item (Task): The Task object to be added to storage.

\item[{Returns:}] \leavevmode
task\_item.key (int): Newly added Task's key.

\end{description}

Raises:

The Task object is given a key and appended to the list of Tasks in
the database.

\end{fulllineitems}


\index{delete() (storage.SQLiteStorage method)}

\begin{fulllineitems}
\phantomsection\label{index:storage.SQLiteStorage.delete}\pysiglinewithargsret{\bfcode{delete}}{\emph{key}}{}
Delete an existing Task in the database storage.
\begin{description}
\item[{Args:}] \leavevmode
key (int): The key for the desired Task object to delete.

\item[{Returns:}] \leavevmode
key\_match (int): Task's key that was deleted in storage.

\end{description}

Raises:

Using the given key, find the matching Task in the database and
delete it. If none is found, nothing is deleted and return None.

\end{fulllineitems}


\index{find() (storage.SQLiteStorage method)}

\begin{fulllineitems}
\phantomsection\label{index:storage.SQLiteStorage.find}\pysiglinewithargsret{\bfcode{find}}{\emph{key=None}}{}
Return a Task given it's key.
\begin{description}
\item[{Args:}] \leavevmode
key (int): The key for the desired Task object.

\item[{Returns:}] \leavevmode
task\_item (Task): Task with matching key.

\end{description}

Raises:

Using the given key, get the Task with the matching key from the
database and return the Task. If none is found return None.

\end{fulllineitems}


\index{get\_all() (storage.SQLiteStorage method)}

\begin{fulllineitems}
\phantomsection\label{index:storage.SQLiteStorage.get_all}\pysiglinewithargsret{\bfcode{get\_all}}{}{}
Return a list of all Task's.

\end{fulllineitems}


\index{search() (storage.SQLiteStorage method)}

\begin{fulllineitems}
\phantomsection\label{index:storage.SQLiteStorage.search}\pysiglinewithargsret{\bfcode{search}}{\emph{search\_task}}{}
Return a Task list given a search Task.
\begin{description}
\item[{Args:}] \leavevmode
search\_task (Task): The Task to be used for searching.

\item[{Returns:}] \leavevmode
task\_search\_list (Task{[}{]}): List of Tasks matching search criteria.

\end{description}

Raises:

Using the given search Task, return a Task list of all Tasks that
match the search Task's attributes.

\end{fulllineitems}


\index{update() (storage.SQLiteStorage method)}

\begin{fulllineitems}
\phantomsection\label{index:storage.SQLiteStorage.update}\pysiglinewithargsret{\bfcode{update}}{\emph{task\_item}}{}
Update an existing Task in the database storage.
\begin{description}
\item[{Args:}] \leavevmode
task\_item (Task): The Task object to be updated.

\item[{Returns:}] \leavevmode
key\_match (int): Task's key that was updating in storage.

\end{description}

Raises:

Using the given Task's key, find the matching Task in the database and
replace it with the given Task then return the old Task. If none is
found, update nothing and return None.

\end{fulllineitems}


\end{fulllineitems}


\index{Storage (class in storage)}

\begin{fulllineitems}
\phantomsection\label{index:storage.Storage}\pysigline{\strong{class }\code{storage.}\bfcode{Storage}}{}
Abstract base class for Task storage.
\begin{description}
\item[{Public methods:}] \leavevmode
add(task\_item)
find(key = None)
get\_all()
update(task\_item)
delete(key)
search(search\_task)

\end{description}

\index{add() (storage.Storage method)}

\begin{fulllineitems}
\phantomsection\label{index:storage.Storage.add}\pysiglinewithargsret{\bfcode{add}}{\emph{task\_item}}{}
This functions is to be overridden by a specific storage method.

\end{fulllineitems}


\index{delete() (storage.Storage method)}

\begin{fulllineitems}
\phantomsection\label{index:storage.Storage.delete}\pysiglinewithargsret{\bfcode{delete}}{\emph{key}}{}
This functions is to be overridden by a specific storage method.

\end{fulllineitems}


\index{find() (storage.Storage method)}

\begin{fulllineitems}
\phantomsection\label{index:storage.Storage.find}\pysiglinewithargsret{\bfcode{find}}{\emph{key=None}}{}
This functions is to be overridden by a specific storage method.

\end{fulllineitems}


\index{get\_all() (storage.Storage method)}

\begin{fulllineitems}
\phantomsection\label{index:storage.Storage.get_all}\pysiglinewithargsret{\bfcode{get\_all}}{}{}
This functions is to be overridden by a specific storage method.

\end{fulllineitems}


\index{search() (storage.Storage method)}

\begin{fulllineitems}
\phantomsection\label{index:storage.Storage.search}\pysiglinewithargsret{\bfcode{search}}{\emph{search\_task}}{}
This functions is to be overridden by a specific storage method.

\end{fulllineitems}


\index{update() (storage.Storage method)}

\begin{fulllineitems}
\phantomsection\label{index:storage.Storage.update}\pysiglinewithargsret{\bfcode{update}}{\emph{task\_item}}{}
This functions is to be overridden by a specific storage method.

\end{fulllineitems}


\end{fulllineitems}


\index{StorageFactory (class in storage)}

\begin{fulllineitems}
\phantomsection\label{index:storage.StorageFactory}\pysigline{\strong{class }\code{storage.}\bfcode{StorageFactory}}{}
Interface for getting a storage instance.
\begin{description}
\item[{Public methods:}] \leavevmode
get(storage\_type, {\color{red}\bfseries{}**}kwargs)

\end{description}

Select the type of storage in which to store Task objects and pass
arguments to the specified storage classes constructor.

\index{get() (storage.StorageFactory static method)}

\begin{fulllineitems}
\phantomsection\label{index:storage.StorageFactory.get}\pysiglinewithargsret{\strong{static }\bfcode{get}}{\emph{storage\_type}, \emph{**kwargs}}{}
Return a Task storage instance.
\begin{description}
\item[{Args:}] \leavevmode
storage\_type (str): Name of the desired storage type.
kwargs (str): Keyword arguments specific to each storage type.

\item[{Returns:}] \leavevmode
storage\_instance (Storage): Child instance of a storage class.

\end{description}

Using the given storage type, create an instance with the given
optional keyword arguments and return the storage instance.

\end{fulllineitems}


\end{fulllineitems}



\chapter{Indices and tables}
\label{index:indices-and-tables}\begin{itemize}
\item {} 
\emph{genindex}

\item {} 
\emph{modindex}

\item {} 
\emph{search}

\end{itemize}


\renewcommand{\indexname}{Python Module Index}
\begin{theindex}
\def\bigletter#1{{\Large\sffamily#1}\nopagebreak\vspace{1mm}}
\bigletter{c}
\item {\texttt{cliparser}}, \pageref{index:module-cliparser}
\indexspace
\bigletter{s}
\item {\texttt{storage}}, \pageref{index:module-storage}
\end{theindex}

\renewcommand{\indexname}{Index}
\printindex
\end{document}
